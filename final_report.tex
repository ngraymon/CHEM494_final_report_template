\documentclass[11pt, letterpaper, oneside]{article}  % define the basic details of our document
% ------------------------------------------------------------------------------------------------------------
% ------------------------------------------------------------------------------------------------------------
\usepackage[utf8]{inputenc}  % Required for inputting international characters
\usepackage[T1]{fontenc} % Output font encoding for international characters
% ------------------------------------------------------------------------------------------------------------
\usepackage[acronym,nomain,xindy]{glossaries} % provides \gls command
% ------------------------------------------------------------------------------------------------------------
\usepackage{amsmath, amssymb, commath}  % standard math packages
% ------------------------------------------------------------------------------------------------------------
\usepackage{graphicx}               % manages images
% ------------------------------------------------------------------------------------------------------------
\usepackage[version=4]{mhchem}      % provides chemical notation formatting (OH^2)
\usepackage{enumitem}               % change the item type of enumerate environmentst
\usepackage{nicefrac}               % slanted fractions
\usepackage{booktabs}               % provides \toprule\midrule\botrule (for tables)
\usepackage{physics}                % provides \bra \ket \dd \pdv commands and more
\usepackage{siunitx}                % provides scientific unit fonts
\usepackage{dsfont}                 % provides \mathds{}
\usepackage{bm}                     % bolding for all args \bm{} command
% ------------------------------------------------------------------------------------------------------------
\usepackage{cleveref}               % really good references
% ------------------------------------------------------------------------------------------------------------
\usepackage[margin=1in]{geometry} % sets the margins to 1 inch
\usepackage{setspace} % provides the \doublespacing command
% ------------------------------------------------------------------------------------------------------------
\usepackage[backend=biber,style=phys,biblabel=brackets]{biblatex}  % this manages references
% ------------------------------------------------------------------------------------------------------------
%%%%%%%%%%%%%%%%      TITLE PAGE     %%%%%%%%%%%%%%%%
\usepackage[british]{datetime2}
% %%%%%%%%%%%%%%%%%%%%%%%%%%%%%%%%
% all of these dimensions assume that the margins are 1 inch
\newlength{\upperverticalspacing} % for my convenience
\newcommand*{\maketitlepage}{\begingroup% Gentle Madness
\upperverticalspacing = 0.01\textheight
\vspace*{\baselineskip}
\vfill
    \hbox{%
    \hspace*{0.2\textwidth}%
    \rule{1pt}{\textheight}
    \hspace*{0.05\textwidth}%
    \parbox[b]{0.75\textwidth}{
    \vbox{%
        \vspace{\upperverticalspacing}
        {\noindent\Huge\bfseries    Title\\[0.5\baselineskip]
                                    of the report\\[0.5\baselineskip]
                                    }\\[1\baselineskip]
        {\Large\itshape Subtitle\\[0.5\baselineskip]
                        of the report}\\[4\baselineskip]
        {\Large Your name}\\[1\baselineskip]
        {\normalsize Supervisor(s): Their name(s)}\\[0.5\baselineskip]
        {\normalsize Submitted in partial fulfillment of CHEM 494}\par
        \vspace{0.5\textheight}
        {\noindent University of Waterloo}\\{\noindent \today }\\[\baselineskip]%\DTMdisplaydate{2016}{03}{18}{-1}
        }% end of vbox
        }% end of parbox
    }% end of hbox
\vfill
\null
\endgroup}
%%%%%%%%%%%%%%%%      TITLE PAGE     %%%%%%%%%%%%%%%% % this provides the \maketitlepage command
% ------------------------------------------------------------------------------------------------------------
\usepackage{xparse}         % to create NewDocumentCommands
% ------------------------------------------------------------------------------------------------------------
% new clever ref
% \crefname{verb}{output}{output}
% ------------------------------------------------------------------------------------------------------------
% macros for basis sets
% \NewDocumentCommand{\sixG}{}{$6\text{-}31\text{G*}$\,}
% \NewDocumentCommand{\sixGpp}{}{$6\text{-}31\text{++G**}$\,}
% %
% \NewDocumentCommand{\ccPVTZ}{}{cc-PVTZ\,}
% \NewDocumentCommand{\TZPP}{}{TZ2P\,}
% \NewDocumentCommand{\DZP}{}{DZP\,}
% ------------------------------------------------------------------------------------------------------------

% \NewDocumentCommand{\ihchi}{ O{} O{} }{\raisebox{\depth}{$#1\chi$}}
% \NewDocumentCommand{\hchi}{}{{\mathpaletter\ihchi\relax}}
\NewDocumentCommand{\hchi}{ G{} G{} }{ \raisebox{\depth}{$\chi$}_{#1}^{#2} }
% ------------------------------------------------------------------------------------------------------------
% commands for position and momenutm operators
\NewDocumentCommand{\p}{G{} G{}}{p_{#1}^{#2}}
\NewDocumentCommand{\q}{G{} G{}}{q_{#1}^{#2}}
\NewDocumentCommand{\hp}{G{} G{}}{\hat{p}_{#1}^{#2}}
\NewDocumentCommand{\hq}{G{} G{}}{\hat{q}_{#1}^{#2}}
% ------------------------------------------------------------------------------------------------------------
% commands for Hamiltonian, kinetic and potential operators
% \RenewDocumentCommand{\H}{ O{} }{\hat{H}_{#1}}
\NewDocumentCommand{\Ham}{ G{} }{\hat{H}_{#1}}
\NewDocumentCommand{\h}{ G{} }{\hat{h}_{#1}}
\NewDocumentCommand{\T}{ G{} }{\hat{T}_{#1}}
\NewDocumentCommand{\V}{ G{} }{\hat{V}_{#1}}
\NewDocumentCommand{\U}{ G{} }{\hat{U}_{#1}}
\NewDocumentCommand{\K}{ G{} }{\hat{K}_{#1}}
% ------------------------------------------------------------------------------------------------------------
% mathematical shorthands
\NewDocumentCommand{\kB}{}{k_\text{B}}
\NewDocumentCommand{\kBT}{}{k_\text{B}T}
\NewDocumentCommand{\identity}{}{\mathds{1}}
\NewDocumentCommand{\limP}{}{\lim_{P \to \infty}}
\NewDocumentCommand{\lsum}{ m G{} }{\sum\limits_{#1}^{#2}}
\NewDocumentCommand{\lprod}{ m G{} }{\prod\limits_{#1}^{#2}}
\NewDocumentCommand{\ePH}{}{e^{-\beta_{P}\hat{H}}}
% ------------------------------------------------------------------------------------------------------------
%%%% vector representation
\NewDocumentCommand{\vect}{m}{\bm{#1}} % BOLD  vectors
% \NewDocumentCommand{\vect}{m}{\vec{#1}} % ARROW vectors
% ------------------------------------------------------------------------------------------------------------
%%%% matrix representation
% \NewDocumentCommand{\matr}{m}{#1} % pure math version
% \NewDocumentCommand{\matr}{m}{\bm{#1}} % ISO complying version
\NewDocumentCommand{\matr}{m}{\bm{\underline{#1}}} % underline
% ------------------------------------------------------------------------------------------------------------
% helps for bra-kets and when there are a lot of exponentials
\NewDocumentCommand{\eB}{m}{e^{-\beta#1}}
\NewDocumentCommand{\eT}{m}{e^{-\tau#1}}
\NewDocumentCommand{\eTh}{ }{\eT{\h}}
\NewDocumentCommand{\eTH}{ }{\eT{\Ham}}
\NewDocumentCommand{\eTV}{ }{\eT{\V}}
\NewDocumentCommand{\etf}{m}{e^{-\frac{\tau}{2}#1}}
\NewDocumentCommand{\etfh}{ }{\etf{\h}}
% ------------------------------------------------------------------------------------------------------------              % where we store the NewDocumentCommands
% ------------------------------------------------------------------------------------------------------------
      % all the document preperation is inside this file
% ------------------------------------------------------------------------------------------------------------
\usepackage{lipsum}  % just to provide filler text for template spacing
\usepackage{mwe}     % just to provide filler images
% ------------------------------------------------------------------------------------------------------------
\addbibresource{references.bib}
\graphicspath{{./images/}} % look inside ./Images relative to the location of the main .tex file
\loadglsentries{./glossary_entries.tex}   % all our acronyms
% \makeglossaries
% ------------------------------------------------------------------------------------------------------------

% Length of report will vary with the project and its details, but normally the main body of the report will range from 15-30 pages. Neatness, logical order of thought, clarity of expression are important--an overly long report is not necessarily a good one. If in doubt, get your supervisor's advice.

% ------------------------------------------------------------------------------------------------------------
% ------------------------------------------------------------------------------------------------------------
\maketitlepage \pagestyle{empty} % this creates the title page
\begin{document}
\clearpage \doublespacing \pagestyle{plain}% specific format for all other pages
% ------------------------------------------------------------------------------------------------------------
\section*{Acknowledgements}
% Acknowledge the assistance of the research supervisor and any other faculty members, graduate students, departmental members and others who have been especially helpful with any aspect of the project.
\lipsum[39]
\newpage%


% ------------------------------------------------------------------------------------------------------------
\section*{Summary}
% The next page should be a Summary, not normally more than half a page, concisely summarizing what is being presented in the report. The Summary will highlight any important features discovered or determined.
Here is some example text.
This is an equation:
\begin{equation}\label{eq:template}
    x \times 2 = x + x.
\end{equation}
Here is an example table:
\begin{table}[h!]
    \centering
    \begin{tabular}{ l c r }
        \toprule
        x & y & z \\
        \midrule
        1 & 2 & 3 \\
        4 & 5 & 6 \\
        7 & 8 & 9 \\
        % \botrule
    \end{tabular}
    \caption{\label{table:one} Computational Parameters.}
\end{table}\\
Here is a reference~\cite{goodfellow1990molecular}, and another reference~\cite{schmidt2014inclusion}, and 3 different references at once~\cite{sarsa2000path,constable2013langevin,goodfellow1990molecular}.
This is how I would refer back to \cref{eq:template}, and how I reference \cref{table:one}.\\
The rest of the document has filler example text using the lipsum package.
This is how you use acronyms/glossary entries: \gls{pi}. Then the next time I talk about \gls{pi}. Another example would be \gls{z}, and we can talk about multiple \glspl{z}, or use the full description of the \glsdesc{z}.
\newpage


% ------------------------------------------------------------------------------------------------------------
\pagestyle{plain}
\pagenumbering{roman} % use roman page numbering
\setcounter{page}{1} % reset the page count to 1
% ------------------------------------------------------------------------------------------------------------
\tableofcontents
\newpage

\listoffigures
\newpage

\listoftables
\newpage


% ------------------------------------------------------------------------------------------------------------
\pagenumbering{arabic} % change to arabic page numbering
\setcounter{page}{1} % reset the page count to 1
% ------------------------------------------------------------------------------------------------------------
\section{Introduction}
\lipsum[39]
\subsection{Outline}
\lipsum[41-42]
\newpage%

% ------------------------------------------------------------------------------------------------------------
\section{Background}
% here are some example sub headings
\subsection{Vibronic coupling}
\lipsum[53-57]
\subsection{Wavefunctions}
\lipsum[58-62]
\subsection{Real time propagation}
\lipsum[62-65]
\newpage%

% ------------------------------------------------------------------------------------------------------------
\section{Theoretical Methods / Proposal}
\subsection{First part of proposal}
\lipsum[47-50]
\subsection{A possible second part of the proposal}
\lipsum[47-50]
\subsection{A third part of the proposal}
\lipsum[60-65]
\newpage%


% ------------------------------------------------------------------------------------------------------------
\section{Results \& Discussion}
\lipsum[41-47]

\begin{figure}[!ht]
    \center
    \includegraphics[width=0.8\columnwidth]{example-image}
    \caption{\label{fig:one}caption of image}
\end{figure}

\lipsum[70-72]

\begin{figure}
    \center
    \includegraphics[width=1.0\linewidth]{example-image}
    \caption{\label{fig:two}caption of another image}
\end{figure}

\lipsum[70-72]

\begin{figure}[!h]
    \center
    \includegraphics[width=0.8\columnwidth]{example-image}
    \caption{\label{fig:three}caption of image}
\end{figure}

\newpage%

% ------------------------------------------------------------------------------------------------------------
\section{Concluding Remarks}
\lipsum[80-83]
\newpage%

% ----------------------------------------------------------------------------------------------------------
\addcontentsline{toc}{section}{References}
\renewcommand*{\bibfont}{\scriptsize}
\printbibliography

% ------------------------------------------------------------------------------------------------------------
\end{document}
% ------------------------------------------------------------------------------------------------------------
% ------------------------------------------------------------------------------------------------------------