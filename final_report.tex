\documentclass[11pt, letterpaper, oneside]{article}  % define the basic details of our document
% ------------------------------------------------------------------------------------------------------------
\input{prelude}      % all the document preperation is inside this file
% ------------------------------------------------------------------------------------------------------------
\usepackage{lipsum}  % just to provide filler text for template spacing
\usepackage{mwe}     % just to provide filler images
% ------------------------------------------------------------------------------------------------------------
\addbibresource{references.bib}
\graphicspath{{./images/}} % look inside ./Images relative to the location of the main .tex file
\loadglsentries{./glossary_entries.tex}   % all our acronyms
% \makeglossaries
% ------------------------------------------------------------------------------------------------------------

% Length of report will vary with the project and its details, but normally the main body of the report will range from 15-30 pages. Neatness, logical order of thought, clarity of expression are important--an overly long report is not necessarily a good one. If in doubt, get your supervisor's advice.

% ------------------------------------------------------------------------------------------------------------
% ------------------------------------------------------------------------------------------------------------
\maketitlepage \pagestyle{empty} % this creates the title page
\begin{document}
\clearpage \doublespacing \pagestyle{plain}% specific format for all other pages
% ------------------------------------------------------------------------------------------------------------
\section*{Acknowledgements}
% Acknowledge the assistance of the research supervisor and any other faculty members, graduate students, departmental members and others who have been especially helpful with any aspect of the project.
\lipsum[39]
\newpage%


% ------------------------------------------------------------------------------------------------------------
\section*{Summary}
% The next page should be a Summary, not normally more than half a page, concisely summarizing what is being presented in the report. The Summary will highlight any important features discovered or determined.
Here is some example text.
This is an equation:
\begin{equation}\label{eq:template}
    x \times 2 = x + x.
\end{equation}
Here is an example table:
\begin{table}[h!]
    \centering
    \begin{tabular}{ l c r }
        \toprule
        x & y & z \\
        \midrule
        1 & 2 & 3 \\
        4 & 5 & 6 \\
        7 & 8 & 9 \\
        % \botrule
    \end{tabular}
    \caption{\label{table:one} Computational Parameters.}
\end{table}\\
Here is a reference~\cite{goodfellow1990molecular}, and another reference~\cite{schmidt2014inclusion}, and 3 different references at once~\cite{sarsa2000path,constable2013langevin,goodfellow1990molecular}.
This is how I would refer back to \cref{eq:template}, and how I reference \cref{table:one}.\\
The rest of the document has filler example text using the lipsum package.
This is how you use acronyms/glossary entries: \gls{pi}. Then the next time I talk about \gls{pi}. Another example would be \gls{z}, and we can talk about multiple \glspl{z}, or use the full description of the \glsdesc{z}.
\newpage


% ------------------------------------------------------------------------------------------------------------
\pagestyle{plain}
\pagenumbering{roman} % use roman page numbering
\setcounter{page}{1} % reset the page count to 1
% ------------------------------------------------------------------------------------------------------------
\tableofcontents
\newpage

\listoffigures
\newpage

\listoftables
\newpage


% ------------------------------------------------------------------------------------------------------------
\pagenumbering{arabic} % change to arabic page numbering
\setcounter{page}{1} % reset the page count to 1
% ------------------------------------------------------------------------------------------------------------
\section{Introduction}
\lipsum[39]
\subsection{Outline}
\lipsum[41-42]
\newpage%

% ------------------------------------------------------------------------------------------------------------
\section{Background}
% here are some example sub headings
\subsection{Vibronic coupling}
\lipsum[53-57]
\subsection{Wavefunctions}
\lipsum[58-62]
\subsection{Real time propagation}
\lipsum[62-65]
\newpage%

% ------------------------------------------------------------------------------------------------------------
\section{Theoretical Methods / Proposal}
\subsection{First part of proposal}
\lipsum[47-50]
\subsection{A possible second part of the proposal}
\lipsum[47-50]
\subsection{A third part of the proposal}
\lipsum[60-65]
\newpage%


% ------------------------------------------------------------------------------------------------------------
\section{Results \& Discussion}
\lipsum[41-47]

\begin{figure}[!ht]
    \center
    \includegraphics[width=0.8\columnwidth]{example-image}
    \caption{\label{fig:one}caption of image}
\end{figure}

\lipsum[70-72]

\begin{figure}
    \center
    \includegraphics[width=1.0\linewidth]{example-image}
    \caption{\label{fig:two}caption of another image}
\end{figure}

\lipsum[70-72]

\begin{figure}[!h]
    \center
    \includegraphics[width=0.8\columnwidth]{example-image}
    \caption{\label{fig:three}caption of image}
\end{figure}

\newpage%

% ------------------------------------------------------------------------------------------------------------
\section{Concluding Remarks}
\lipsum[80-83]
\newpage%

% ----------------------------------------------------------------------------------------------------------
\addcontentsline{toc}{section}{References}
\renewcommand*{\bibfont}{\scriptsize}
\printbibliography

% ------------------------------------------------------------------------------------------------------------
\end{document}
% ------------------------------------------------------------------------------------------------------------
% ------------------------------------------------------------------------------------------------------------